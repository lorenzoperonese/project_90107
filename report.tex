\documentclass{article}
\usepackage{booktabs}

\usepackage{graphicx}
\usepackage{tabularx}

\title{Gestione servizio di vehicle sharing}

\author{Lorenzo Peronese, Emanuele Argonni}

\begin{document}

\maketitle

\tableofcontents

\newpage

\section{Analisi dei requisiti}

\subsection{Requisiti in linguaggio naturale}

Si intende realizzare una base di dati per la gestione di un servizio di vehicle sharing elettrici, che offra agli utenti la possibilità di noleggiare automobili, biciclette, monopattini e scooter.
Il sistema dovrà gestire tutti gli aspetti relativi al parco veicoli, agli utenti, alle prenotazioni e ai pagamenti.

Per ciascun veicolo, identificato da un codice univoco, si vogliono memorizzare: tipologia (auto, bici, scooter, monopattino), marca, modello, targa, ultima posizione GPS, stato attuale (disponibile, in uso, in ricarica, fuori servizio), numero di posti, livello di carica della batteria, chilometraggio totale, dati dell’assicurazione, data dell’ultima revisione. Inoltre si desidera tenere traccia degli interventi di manutenzione, registrando tipo di intervento, data, costo e officina di riparazione.

Per ogni utente registrato al servizio si desidera tracciare: ID univoco, dati anagrafici (nome, cognome, codice fiscale, data di nascita), numero di telefono, indirizzo email, estremi della patente di guida se necessari (numero, data di scadenza, categoria), dati per la fatturazione, storico dei noleggi effettuati, stato del profilo (attivo, bloccato, in fase di verifica), data di registrazione, metodi di pagamento, ID di abbonamenti attivi.

Riguardo alle stazioni di ricarica dei veicoli si vuole rappresentare: posizione GPS, stato corrente (libera, occupata, in manutenzione, fuori servizio) e tipologia di veicolo che può ricaricare.

Per ogni noleggio, è necessario registrare: ID univoco, ID dell’utente che lo effettua, ID del veicolo utilizzato, data di inizio noleggio, data di fine noleggio, posizione GPS di inizio e fine noleggio, chilometri percorsi, durata del noleggio, costo totale, esito del pagamento.

Si vuole modellare la struttura tariffaria del servizio con tariffe base a tempo per ciascuna tipologia di veicolo e la possibilità di acquistare abbonamenti (giornalieri, settimanali, mensili).

\subsection{Glossario dei termini}

\begin{table}
\centering
\begin{tabularx}{\textwidth}{|X|X|X|X|}
\hline
\textbf{Termine} & \textbf{Descrizione} & \textbf{Sinonimi} & \textbf{Collegamenti} \\ \hline
Parco veicoli & Insieme dei veicoli facenti parte del serivizo & Flotta veicoli & - \\ \hline
Veicolo & Mezzo di trasporto elettrico gestito dal servizio & Mezzo & - \\ \hline
Cliente & Persona registrata al servizio di sharing & Utente, Consumatore & - \\ \hline
Noleggio & Utilizzo di un veicolo per un certo periodo da parte di un cliente & Corsa, Utilizzo & - \\ \hline
Manutenzione & Intervento tecnico effettuato su un veicolo & Riparazione, Intervento & - \\ \hline
Centro di ricarica & Luogo dotato di colonnine per la ricarica dei veicoli & - & - \\ \hline
Stazione di ricarica & Punto fisico in cui ricaricare i veicoli & Colonnina di ricarica & - \\ \hline
Stato utente & Condizione attuale del profilo utente (attivo, sospeso, in stato di verifica) & Stato profilo, stato account & - \\ \hline
Abbonamento & Piano tariffario prepagato (giornaliero, mensile, ecc.) & Pass & - \\ \hline
Tariffa base & Costo del servizio per unità di tempo, differenziato per tipologia di veicolo & Prezzo, Costo & - \\ \hline
Patente di guida & documento necessario per il noleggio di auto e scooter & - & - \\ \hline
Dati di fatturazione & Informazioni dell'utente per ricevuta/fattura & - & - \\ \hline
Dati di pagamento & Informazioni bancarie per saldare il costo del servizio & - & - \\ \hline
Posizione GPS & Coordinate GPS che indicano la posizione di un Veicolo e/o Stazione di ricarica & Coordinate & - \\ \hline
Stato veicolo & Condizione attuale di un veicolo (disponibile, in uso, in ricarica, fuori servizio) & Disponibilità veicolo & - \\ \hline
Stato stazione & Condizione attuale di una stazione di ricarica (libera, occupata, in manutenzione, fuori servizio) & - & - \\ \hline
Officina di riparazione & Struttura che effettua manutenzione dei veicoli & Centro assistenza & - \\ \hline


\end{tabularx}
\caption{Glossario dei termini}
\label{table_glossario_termini}
\end{table}

\subsection{Eliminazione delle ambiguità}

\subsection{Struttura dei requisiti}

\subsection{Specifica operazioni}

\newpage % tmp

\begin{enumerate}
    \item \textbf{Veicolo:} 
    \begin{itemize}
        \item \textsc{inserimento:} aggiunta di un veicolo alla flotta
        \item \textsc{modifica:} aggiornamento dei dati dello stato del veicolo (livello batteria, posizione GPS, km totali, manutenzioni) 
        \item \textsc{cancellazione:} rimozione di un veicolo dalla flotta
        \item \textsc{ricerca:} visualizzazione dei dettagli di uno o più veicoli per ID, targa, tipologia, livello di batteria, posizione
    \end{itemize}
    \item \textbf{Noleggio:}
    \begin{itemize}
        \item \textsc{inserimento:} avvio di un nuovo noleggio da parte di un cliente
        \item \textsc{modifica:} aggiornamento dei dati (km percorsi, costo, durata) alla conclusione del noleggio
        \item \textsc{ricerca:} visualizzare lo storico dei noleggi per utente, per veicolo, per periodo
    \end{itemize}
    \item \textbf{Clienti:}
    \begin{itemize}
        \item \textsc{inserimento:} registrazione di un nuovo utente al servizio
        \item \textsc{modifica:} aggiornamento dei dati di pagamento, stato del profilo
        \item \textsc{cancellazione:} rimozione dell'account di un utente
        \item \textsc{ricerca:} visualizzazione dettagli di un utente per ID, dati anagrafici, email
    \end{itemize}
    \item \textbf{Manutenzioni:}
    \begin{itemize}
        \item \textsc{inserimento:} registrazione di un intervento tecnico per un veicolo
        \item \textsc{ricerca:} visualizzazione lo storico di interventi per un veicolo
    \end{itemize}
    \item \textbf{Centri di ricarica:}
    \begin{itemize}
        \item \textsc{inserimento:} aggiunta di un nuovo centro di ricarica
        \item \textsc{cancellazione:} rimozione di un centro di ricarica non più di competenza del servizio
        \item \textsc{ricerca:} visualizzazione di un centro di ricarica in base a ID, zona geografica
    \end{itemize}
    \item \textbf{Stazioni di ricarica:}
    \begin{itemize}
        \item \textsc{inserimento:} aggiunta di una nuova stazione di ricarica
        \item \textsc{modifica:} aggiornamento dello stato della colonnina
        \item \textsc{cancellazione:} rimozione di una stazione non più di competenza del servizio
        \item \textsc{ricerca:} visualizzazione di colonnine di ricarica in base a ID, zona geografica, centro di ricarica di cui fanno parte
    \end{itemize}
    \item \textbf{Tariffa:}
    \begin{itemize}
        \item \textsc{inserimento:} definizione della tariffa base per tipologia di veicolo
        \item \textsc{modifica:} aggiornamento del prezzo di una tariffa base
        \item \textsc{ricerca:} visualizzazione della tariffa base per tipologia di veicolo
    \end{itemize}
\end{enumerate}

\subsubsection{Cancellazioni}

\subsection{Ricerche}


\section{Progettazione concettuale}

\subsection{Identificazione delle entità e relazioni}

\subsection{Un primo scheletro dello schema}

\subsection{Sviluppo delle componenti dello scheletro}

\subsection{Unione delle componenti nello schema finale ridotto}

\subsection{Dizionario dei dati}

\subsection{Regole aziendali}

\section{Progettazione logica}

\subsection{Tavole dei volumi e delle operazioni}

\subsection{Ristrutturazione dello schema concettuale}

\subsection{Normalizzazione}

\subsection{Traduzione verso il modello relazionale}

\section{Codifica SQL}

\subsection{Definizione dello schema}

\subsection{Codifica delle operazioni}

\section{Testing}

\end{document}
